\documentclass{article}

\usepackage{float}
\usepackage{placeins}

\def\code#1{\texttt{#1}}

\begin{document}

\title{\vspace{-3.3cm}The Buddy Algorithm \\
\medskip
\large
ID1206 Operativsystem }
\author{Max Körlinge}
\date{\today}
\maketitle


\section{Background}

Managing memory is one of the operating system's most important duties and of course as such the way it is implemented in different operating systems have seen many changes and improvements over the years. The issue at hand is how to most efficiently divide, hand out, and free, memory space of different sizes, since different programs will need different amounts of memory.

At its core most implementations think of memory in terms of blocks. These blocks are distributed to those who ask for it. To do this, you need some sort of data structures to keep track of what is going on, and algorithms to handle dividing, handing out, and freeing memory. When freeing memory, you also have to \textit{coalesce} it, that means that if space is freed next to more free space, both free spaces should be combined into one large block of free space.

When constructing these data structures and algorithms there is one frequent problem, and that is fragmented memory. This means, in essence, that you are not able to use all available memory, because of inefficient division and/or allocation of memory blocks. For example, if a program wants 8 bytes of memory, and the operating system gives them a block of 16 bytes, no other program can use the excess 8 bytes that are not used in that block of memory. This phenomena is called \textit{fragmentation} and it comes in two ways: internal and external fragmentation. Internal fragmentation is the example just described, where a used memory block contains free memory. External fragmentation occurs when there is free space in between distributed memory blocks that is too small to be used by other programs.

One strategy for allocating memory is the \textit{Buddy Algorithm}, or buddy allocation, whose main strength is coalescing, described above. In this strategy, when a request for memory is made, the algorithm divides free memory in blocks of two until reaching the nearest available block that the requested memory fits in and returns it to the user. When freeing the memory, the algorithm will check to see if the "buddy", the other free block that was created in the last split, is also free, and if it is, the algorithm will merge the buddies together, and keep doing that until the buddy is not free.

It is not immediately obvious why coalescing is fast using this strategy until you realise that finding the buddy of a memory block is very easy, because the memory address of the buddies only differ by one bit. This bit is located in the address at a spot determined by how large the memory block is, that is, at what level in the binary tree of blocks it is located. This all means that if you know the address of the block, and what level it is at, you only have to shift one bit to find the buddy of the block.

Now, for each block to know these things about themselves (whether it is taken, and at what level it is), it is necessary for each block to contain information about itself. Thus, each block contains a header where such information is stored, and when a user gets access to a memory block, this header is hidden from the user.

In this assignment, we were given the task of implementing a memory allocating library similar to \textit{malloc}, which uses the buddy algorithm. We received sample code and a walk-through of the methods present in the sample code, and were instructed to implement the two main algorithms: \code{find()}, which is called when requesting a new block, handling the division of the free memory and finding of a suitable block, and \code{insert()}, which is called when freeing memory and handles the coalescing of blocks. We were then asked to write a benchmarking program and run some benchmarks regarding the speed of these operations.

\section{Implementation}

\subsection{Data Structures}

There are two main data structures used in this implementation. The first is a \textit{free list}, an array, with one slot for each level in the binary memory tree. That is, depending on the maximum and minimum size of a memory block, there will be a certain amount of possible levels from the largest block to the smallest block. Each level will have one slot in the array.

The second data structure is the \textit{block header}, which is contained in each memory block, and has information on whether it is taken or free, what level it is on, and it is also part of a double-linked list. The double-linked list is necessary because there may be many free blocks at a certain level. This means that at each slot in the free list, you find the first block in a double-linked list of free blocks at that level.

\subsection{find()}

Our strategy in pseudo code:

\noindent \code{	find(int level) \{  \\
\hspace*{2em}	//find block at this level in free list, if not there, move up a level and check free list again \\
\hspace*{2em}	//take first free block in list \\
\hspace*{2em}	//make free list at that level point to next block in double-linked list at that level, or NULL if there is none \\
\hspace*{2em} while (i != level) \{ //loop until at requsted level\\
\hspace*{4em}        i--; //step down one level (we are splitting blocks), i is set to the level we found starting free block at \\
\hspace*{4em}        freeblock->level = i; \\
\hspace*{4em}        buddyToInsert = buddy(freeblock); //buddy to be free \\
\hspace*{4em}        buddyToInsert->status = Free; \\
\hspace*{4em}        buddyToInsert->level = i; \\
\hspace*{4em}        buddyToInsert->next = freelist[i]; //buddy will be first in list on level i \\
\hspace*{4em}        if (freelist[i] != NULL) \{  \\
\hspace*{4em}            freelist[i]->prev = buddyToInsert; //previous first block will have buddy as previous \\
\hspace*{4em}      \} \\
\hspace*{4em}        freelist[i] = buddyToInsert; //buddy is inserted into list \\
\hspace*{2em}    \} \\
\hspace*{2em}    freeblock->status = Taken; \\
\hspace*{2em}	return freeblock; \\
	\}
}

Full implementation in C is at \code{https://github.com/fongie/malloc/blob/master/buddy.c\#L102}

\subsection{insert()}
Our strategy in pseudo code:

\noindent \code{
	insert(struct head *block) \{ \\
\hspace*{2em}	level = block->level //which level are you? \\
\hspace*{2em}	buddy = buddy(block) //who is your buddy? \\
\hspace*{2em}	if buddy is free \{ \\
\hspace*{4em}		mergedBlock = merge(block,buddy) \\
\hspace*{4em}		mergedBlock->level = level+1 \\
\hspace*{4em}		removeFromFreeList(buddy) \\
\hspace*{4em}		insert(mergedBlock) //recursive to keep merging until buddy not free \\
\hspace*{4em}		return \\
\hspace*{2em}	\} else \{ \\
\hspace*{4em}		finalInsertInFreeList(block) //place block in free list \\
\hspace*{2em}	\} \\
	\}
}

Full implementation in C is at \code{https://github.com/fongie/malloc/blob/master/buddy.c\#L185}
\FloatBarrier
\section{Results}

\begin{table}[]
\caption{0\% allocated compared to number of requested blocks}
\begin{tabular}{llll}
	size         & 2-4096     & 3000-4096   & 2-100         \\
	balloc       & 0.00000168 & 0.000000426 & 0.000000553  \\
	bfree        & 0.00000162 & 0.000000400 & 0.000000568 \\
\end{tabular}
\end{table}

\begin{table}[]
\caption{25\% allocated compared to number of requested blocks}
\begin{tabular}{llll}
size         & 2-4096     & 3000-4096   & 2-100         \\
balloc       & 0.000000000556 & 0.000000000426 & 0.000000000457  \\
bfree        & 0.000000000401 & 0.000000000523 & 0.000000000396  \\
\end{tabular}
\end{table}

\begin{table}[]
\caption{75\% allocated compared to number of requested blocks}
\begin{tabular}{llll}
size         & 2-4096     & 3000-4096   & 2-100         \\
balloc       & 0.000000000442 & 0.000000000471 & 0.000000000458 \\
bfree        & 0.000000000379 & 0.000000000408 & 0.000000000377 \\
\end{tabular}
\end{table}

We wrote a benchmarking program at \code{https://github.com/fongie/malloc/blob/master/bench.c}. It studies how the algorithm performs under various circumstances, such as changing the size of the requested blocks, and whether there are already allocated blocks or not. To do this, we time the \code{balloc} and \code{bfree} operations 1000 times and return an average operation time, while changing these circumstances. You can find the results in Table 1, 2, and 3.

\section{Discussion}

The results show that the operation is at its slowest when allocating a block of memory for the first time. This is probably due to the fact that the first time a block of memory is allocated, the program has to build the binary tree and free list. When freeing the only block allocated, it is also slow, because it has to coalesce the entire binary tree all the way to the top.

When some blocks have already been allocated, these algorithms do not have to work all the way to the top or the bottom, instead, the \code{balloc} algorithm can find a block at the requested level right away in the free list and return that, or the \code{bfree} algorithm could find when freeing a block that its buddy is taken and simply put itself in the free list, not performing any, or not so many, coalescing operations.

Regarding size for \code{balloc}, in two out of three different allocations the bigger sizes are handled faster than smaller ones. This makes sense for when no blocks are allocated and when few are, because when you ask for larger sizes the algorithm can only go so far up in level before it either has to find a free block, or no free block. For \code{bfree}, results are the opposite. This is strange, because that algorithm should also work faster for larger blocks, for the same reason as above, that for larger blocks it never has to merge the maximum number of blocks, while it might happen for smaller blocks.

The results are mostly as expected, and where they are not, the benchmarking program can probably be blamed. The benchmarking program could be improved to more randomly select which blocks to free, by for example storing allocated blocks in a list. Due to time constraints this was not implemented, which means that when freeing blocks, the double-linked list does not have be traversed in the \code{bfree} function, when removing a block from the free list. This operation should be quite costly, probably the most costly because it is the only time we are traversing a list, not just handling the head of the list. This might not be reflected in the results that we got.

\end{document}